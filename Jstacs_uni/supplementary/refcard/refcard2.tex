\documentclass[10pt]{scrartcl}

\usepackage[landscape]{geometry}
\usepackage{multicol}
\usepackage{mdwlist}
\usepackage{ exscale, latexsym, amsthm, amssymb, amsmath,hyperref,color, ifthen}
\usepackage[ansinew]{inputenc}
\usepackage{xspace}

\hypersetup{
	pdfpagemode = None, pdfpagelayout = OneColumn, pdfstartpage = 1, pdfstartview = FitH,
	colorlinks = true, urlcolor = blue, linkcolor = black, citecolor = black
}

\setlength{\textwidth}{150mm}
\setlength{\oddsidemargin}{10mm}

\renewcommand{\thesection}{\arabic{section}}

\newcommand{\jstacs}{\textsc{Jstacs}}
\newcommand{\APIhome}{http://www.jstacs.de/api-2.0/}
\newcommand{\link}[2]{\href{\APIhome/de/jstacs/#1/#2.html}{#2}}

%concrete links
\newcommand{\SubclassFinder}{\link{utils}{SubclassFinder}}
\newcommand{\Singleton}{\link{}{Singleton}}
\newcommand{\Storable}{\link{}{Storable}}
\newcommand{\XMLParser}{\link{io}{XMLParser}}
\newcommand{\Parameter}{\link{parameters}{Parameter}}
\newcommand{\SimpleParameter}{\link{parameters}{SimpleParameter}}
\newcommand{\EnumParameter}{\link{parameters}{EnumParameter}}
\newcommand{\SelectionParameter}{\link{parameters}{SelectionParameter}}
\newcommand{\FileParameter}{\link{parameters}{FileParameter}}
\newcommand{\RangeParameter}{\link{parameters}{RangeParameter}}
\newcommand{\NumberValidator}{\link{parameters/validation}{NumberValidator}}
\newcommand{\ParameterSet}{\link{parameters}{ParameterSet}}
\newcommand{\InstanceParameterSet}{\link{parameters}{InstanceParameterSet}}
\newcommand{\SimpleParameterSet}{\link{parameters}{SimpleParameterSet}}
\newcommand{\SequenceScoringParameterSet}{\link{parameters}{SequenceScoringParameterSet}}

\newcommand{\Result}{\link{results}{Result}}
\newcommand{\NumericalResult}{\link{results}{NumericalResult}}
\newcommand{\CategoricalResult}{\link{results}{CategoricalResult}}
\newcommand{\ResultSet}{\link{results}{ResultSet}}
\newcommand{\MeanResultSet}{\link{results}{MeanResultSet}}
\newcommand{\NumericalResultSet}{\link{results}{NumericalResultSet}}
\newcommand{\ListResult}{\link{results}{ListResult}}

\newcommand{\DNAAlphabet}{\link{data/alphabets}{DNAAlphabet}}
\newcommand{\ComplementableDiscreteAlphabet}{\link{data/alphabets}{ComplementableDiscreteAlphabet}}
\newcommand{\SparseSequence}{\link{data/sequences}{SparseSequence}}
\newcommand{\DiscreteAlphabet}{\link{data/alphabets}{DiscreteAlphabet}}
\newcommand{\Alphabet}{\link{data/alphabets}{Alphabet}}
\newcommand{\AlphabetContainer}{\link{data}{AlphabetContainer}}
\newcommand{\Sequence}{\link{data/sequences}{Sequence}}
\newcommand{\DataSet}{\link{data}{DataSet}}
\newcommand{\DNADataSet}{\link{data}{DNADataSet}}
\newcommand{\PermutedSequence}{\link{data/sequences}{PermutedSequence}}
\newcommand{\SequenceAnnotation}{\link{data/sequences/annotation}{SequenceAnnotation}}
\newcommand{\SequenceAnnotationParser}{\link{data/sequences/annotation}{SequenceAnnotationParser}}
\newcommand{\SplitSequenceAnnotationParser}{\link{data/sequences/annotation}{SplitSequenceAnnotationParser}}
\newcommand{\SimpleSequenceAnnotationParser}{\link{data/sequences/annotation}{SimpleSequenceAnnotationParser}}
\newcommand{\StringExtractor}{\link{io}{StringExtractor}}
\newcommand{\SparseStringExtractor}{\link{io}{SparseStringExtractor}}
\newcommand{\TrainSM}{\link{sequenceScores/statisticalModels/trainable}{TrainableStatisticalModel}}
\newcommand{\TrainSMFactory}{\link{sequenceScores/statisticalModels/trainable}{TrainSMFactory}}
\newcommand{\AbstractTrainSM}{\link{sequenceScores/statisticalModels/trainable}{AbstractTrainSM}}
\newcommand{\StrandTrainSM}{\link{sequenceScores/statisticalModels/trainable/mixture}{StrandTrainSM}}
\newcommand{\StrandDiffSM}{\link{sequenceScores/statisticalModels/differentiable/mixture}{StrandDiffSM}}
\newcommand{\IndependentProductDiffSM}{\link{sequenceScores/statisticalModels/differentiable}{IndependentProductDiffSM}}


\newcommand{\HomogeneousModel}{\link{sequenceScores/statisticalModels/trainable/discrete/homogeneous}{HomogeneousModel}}
\newcommand{\HMMFactory}{\link{sequenceScores/statisticalModels/trainable/hmm}{HMMFactory}}
\newcommand{\HigherOrderHMM}{\link{sequenceScores/statisticalModels/trainable/hmm/models}{HigherOrderHMM}}
\newcommand{\TransitionElement}{\link{sequenceScores/statisticalModels/trainable/hmm/transitions/elements}{TransitionElement}}


\newcommand{\DiffSM}{\link{sequenceScores/statisticalModels/differentiable}{DifferentiableStatisticalModel}}
\newcommand{\DiffSS}{\link{sequenceScores/differentiable}{DifferentiableSequenceScore}}
\newcommand{\AbstractDiffSS}{\link{sequenceScores/differentiable}{AbstractDifferentiableSequenceScore}}
\newcommand{\AbstractDiffSM}{\link{sequenceScores/statisticalModels/differentiable}{AbstractDifferentiableStatisticalModel}}


\newcommand{\SeqScore}{\link{sequenceScores}{SequenceScore}}
\newcommand{\StatMod}{\link{sequenceScores/statisticalModels}{StatisticalModel}}

\newcommand{\AbstractClassifier}{\link{classifiers}{AbstractClassifier}}
\newcommand{\AbstractScoreBasedClassifier}{\link{classifiers}{AbstractScoreBasedClassifier}}
\newcommand{\TrainSMBasedClassifier}{\link{classifiers/trainSMBased}{TrainSMBasedClassifier}}
\newcommand{\GenDisMixClassifier}{\link{classifiers/differentiableSequenceScoreBased/gendismix}{GenDisMixClassifier}}

\newcommand{\PerformanceMeasureParameterSet}{\link{classifiers/performanceMeasures}{PerformanceMeasureParameterSet}}
\newcommand{\NumericalPerformanceMeasureParameterSet}{\link{classifiers/performanceMeasures}{NumericalPerformanceMeasureParameterSet}}
\newcommand{\AbstractPerformanceMeasure}{\link{classifiers/performanceMeasures}{AbstractPerformanceMeasure}}

\newcommand{\ClassifierAssessment}{\link{classifiers/assessment}{ClassifierAssessment}}
\newcommand{\KFoldCrossValidation}{\link{classifiers/assessment}{KFoldCrossValidation}}
\newcommand{\RepeatedHoldOutExperiment}{\link{classifiers/assessment}{RepeatedHoldOutExperiment}}
\newcommand{\KFoldCrossValidationAssessParameterSet}{\link{classifiers/assessment}{KFoldCrossValidationAssessParameterSet}}

\newcommand{\NumericalDifferentiableFunction}{\link{algorithms/optimization}{NumericalDifferentiableFunction}}
\newcommand{\DifferentiableFunction}{\link{algorithms/optimization}{DifferentiableFunction}}
\newcommand{\TerminationCondition}{\link{algorithms/optimization/termination}{TerminationCondition}}

\newcommand{\CombinedCondition}{\link{algorithms/optimization/termination}{CombinedCondition}}

\newcommand{\Costs}{\link{algorithms/alignment/cost}{Costs}}
\newcommand{\Alignment}{\link{algorithms/alignment}{Alignment}}
\newcommand{\AlignmentType}{\link{algorithms/alignment}{Alignment.AlignmentType}}

\newcommand{\REnvironment}{\link{utils}{REnvironment}}
\newcommand{\ArrayHandler}{\link{io}{ArrayHandler}}


%finding the line numer of a specific line in the \codefile
\newcommand{\lineIndex}[1]{
  \setboolean{next}{true}
  \setcounter{off}{0}

  \openin\File=\codefile
  \whiledo{\boolean{next}}{%\and{\boolean{next}}{\not\equal{\FileLine}{#1}}}{
    \ReadNextLine{\File}
    \ifthenelse{\boolean{next}}{\\stepcounter{off}}{}
  }
  \closein\File
  \codefile: \arabic{off}
}

\newboolean{next}
\newcommand{\FileLine}{}
\newread\File

\newcommand{\ReadNextLine}[1]{
  \ifthenelse{\boolean{next}}{
    \read#1 to \FileLine
    \ifeof#1\setboolean{next}{false}
    \else % if last line already is read, EOF appears here
    \fi
  }{}
}
%\leftskip 0.1in
%\parindent -0.1in
\newcommand{\entry}[3]{{\item[]\bfseries #1#2}: #3}
\newcommand{\entrys}[3]{\item[\emph{static}] {\bfseries {#1#2}}: #3}
\newcommand{\entryn}[3]{\item[new] {\bfseries {#1#2}}: #3}

\newcommand{\sep}{\\~\vspace{-0.1cm}}

\geometry{a4paper,left=5mm,right=5mm, top=5mm, bottom=1cm}
\begin{document}
\thispagestyle{empty}
\twocolumn[{\begin{center}\Huge\sfb Jstacs reference card\end{center}}]


\renewcommand{\section}[1]{{
~\vspace{-0.2cm}

\large\sfb #1\\}


}
\begin{flushleft}
%\begin{multicols}{3}
\footnotesize
\section{Data handling}
\begin{itemize*}
\entry{\DNAAlphabet}{.SINGLETON}{Singleton instance of a DNA-alphabet}

\entryn{\DiscreteAlphabet}{(caseInsensitive,alphabet)}{Create an arbitrary discrete alphabet}

\entryn{\ContinuousAlphabet}{(min,max)}{Create a continuous alphabet between min and max}

\entry{\DNAAlphabetContainer}{.SINGLETON}{Singleton instance of aggregate DNA-alphabet}

\entryn{\AlphabetContainer}{(alphabets)}{Create a set of \Alphabet s assigned to positions}\sep

\entrys{\Sequence}{.create(alphabets,string)}{Create a sequence from a string}

\entry{\Sequence}{.getLength()}{Obtain the length of a sequence}

\entry{\Sequence}{.discreteVal(pos)}{Obtain the discrete value at a position (counting from 0) of a sequence}

\entry{\Sequence}{.continuousVal(pos)}{Obtain the continuous value at a position (counting from 0) of a sequence}\sep

\entryn{\DataSet}{(annotation,sequences)}{Create a data set from sequences}

\entryn{\DNADataSet}{(filename)}{Create a data set of DNA sequences from a FastA file}\sep

\entry{\DataSet}{.getNumberOfElements()}{Obtain the number of sequences in a data set}

\entry{\DataSet}{.getElementAt(index)}{Obtain the sequence at index from a data set}

\entry{\DataSet}{.getInfixDataSet(start,length)}{Get a data set containing all infixes of a given length starting at a given position of all sequences in the current data set}
\end{itemize*}

\section{Statistical models}

\begin{itemize*}

\entry{\StatMod}{}{Interface for all statistical models}

\entry{\TrainSM}{}{Interface for statistical models that are trained from a single data set}

%\entry{\AbstractTrainSM}{}{Abstract class for statistical models that are trained from a single data set}

\entry{\DiffSM}{}{Interface for statistical models that can be trained using gradient-based methods}\sep

%\entry{\AbstractDiffSM}{}{Abstract class for statistical models that can be trained using gradient-based methods}

\entry{\TrainSMFactory}{}{Factory for standard implementations of \TrainSM s}

\entrys{\TrainSMFactory}{.createPWM(alphabets,length,ess)}{Create a PWM model of a given length}

\entrys{\TrainSMFactory}{.createInhomogeneousMarkovModel(alphabets,length,\\ess,order)}{Create an inhomogeneous Markov model of a given length and order}

\entrys{\TrainSMFactory}{.createHomogeneousMarkovModel(alphabets,ess,\\order)}{Create a homogeneous Markov model of a given order}

\entrys{\TrainSMFactory}{.createMixtureModel(hyperpars,models)}{Create a mixture model from \TrainSM s}\sep

\entry{\DiffSMFactory}{}{Factory for standard implementations of \DiffSM s}

\entrys{\DiffSMFactory}{.createPWM(alphabets,length,ess)}{Create a PWM model of a given length}

\entrys{\DiffSMFactory}{.createInhomogeneousMarkovModel(alphabets,\\length,ess,order)}{Create an inhomogeneous Markov model of a given length and order}

\entrys{\DiffSMFactory}{.createHomogeneousMarkovModel(alphabets,ess,\\order,priorLength)}{Create a homogeneous Markov model of a given order}

\entrys{\DiffSMFactory}{.createMixtureModel(models)}{Create a mixture model from \DiffSM s}\sep

\entry{\HMMFactory}{}{Factory for standard implementations of hidden Markov models}\sep

\entry{\StatMod}{.emitDataSet(number,length)}{Generate a given number of sequences with specified length from the model using the current parameter values}

\entry{\StatMod}{.getLogProbFor(sequence)}{Obtain the log probility (likelihood) of a sequence for a given model}

\entry{\TrainSM}{.train(data)}{Train a \TrainSM~from a data set}

\entry{\DiffSM}{.initializeFunctionRandomly()}{Initialize the parameters of this model randomly}

\entry{\DiffSM}{.getLogScoreFor(sequence)}{Obtain a log score (typically proportional to the log-likelihood) of a sequence for a given model}

\entry{\DiffSM}{.getLogScoreAndPartialDerivation(sequence,indices,\\partialDers)}{Compute the partial derivations wrt. all parameters for the given sequences and store the parameter indexes and corresponding partial derivations in given lists}

\end{itemize*}

\section{Classifiers}

\begin{itemize*}

\entry{\AbstractClassifier}{}{Abstract class of a classifier}

\entryn{\TrainSMBasedClassifier}{(models)}{Create a classifier from \TrainSM s that is learned by ML or MAP}

\entryn{\MSPClassifier}{(params,prior,models)}{Create a classifier from \DiffSM s that is learned by MCL or MSP}

\entryn{\GenDisMixClassifier}{(params,prior,learnPrinc,models)}{Classifier that learns \DiffSM s using a unified learning principle}\sep

\entry{\AbstractClassifier}{.train(dataSets)}{Train a classifier from training data sets}

\entry{\AbstractClassifier}{.classify(sequence)}{Classify a sequence}

\entry{\AbstractClassifier}{.evaluate(performanceMeasures,exc,dataSet)}{Evaluate performance measures for a given classifier on test data sets}\sep

\entry{\AbstractPerformanceMeasure}{}{Abstract class of all performance measures}

\entryn{\NumericalPerformanceMeasureParameterSet}{()}{Create a set of scalar standard performance measures that are applicable to two-class problems (binary classification)}

\entryn{\PerformanceMeasureParameterSet}{(measures)}{Create a set of performance measures}

\entrys{\PerformanceMeasureParameterSet}{.createFilledParameters()}{Create a set of scalar standard performance measures for binary classification problems that can be used instantly}

\end{itemize*}

\section{Utilities}

\begin{itemize*}

\entry{\Storable}{}{Interface of objects that can be stored to XML}

\entrys{\XMLParser}{.appendObjectWithTags(buffer,storable,tag)}{Append storable object to StringBuffer with given tags}

\entrys{\XMLParser}{.extractObjectForTags(buffer,tag)}{Extract storable object within tags from StringBuffer}\sep

\entryn{\Alignment}{(type,costs)}{Create an object for alignments of sequences}

\entry{\Alignment}{.getAlignment(seq1,seq2)}{Align two sequences}\sep

\entrys{\ArrayHandler}{.clone(array)}{Deep clone a multi-dimensional array}

\entrys{\ArrayHandler}{.createArrayOf(template,num)}{Create an array containing num clones of a template}\sep

\entrys{\ToolBox}{.sum(doubles)}{Compute the sum of all elements in an array}

\entrys{\ToolBox}{.getMaxIndex(doubles)}{Get the index of the maximum value in an array}\sep

\entrys{\Normalisation}{.getLogSum(doubles)}{Compute the logarithm of a sum of values given as logs}

\entrys{\Normalisation}{.sumNormalisation(double)}{Normalize a given array to probabilities}

\end{itemize*}

\end{flushleft}
%\end{multicols}
\end{document}