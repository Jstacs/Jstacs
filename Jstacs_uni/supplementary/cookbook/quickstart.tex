\section{Quick start: Jstacs in a nutshell}\label{start}
\renewcommand{\codefile}{recipes/TrainPWM.java}
This section is for the unpatient who like to directly start using Jstacs without reading the complete cookbook. If you do not belong to this group, you can skip this section.

Here, we provide code snippets for simple tasks including reading a data set or creating models and classifiers that might be frequently used. In addition, some of the basic code examples in section~\nameref{recipes} may also serve as a basis for a quick start into Jstacs.

For reading a FastA file, we call the constructor of the \DNADataSet~with the (absolute or relative) path to the FastA file. Subsequently, we can determine the alphabets used.
\setcounter{off}{37}
\code{1}
Detailed information about data sets, sequences, and alphabets can be found in section~\ref{data}.

\subsection{Statistical models and classifiers using generative learning principles}

In Jstacs, statistical models that use generative learning principles to infer their parameters implement the interface \TrainSM. For convenience, Jstacs provides the \TrainSMFactory, which allows creating various simple models in an easy manner. Creating for instance a position weight matrix (PWM) model is just one line of code.
\addtocounter{off}{3}
\code{0}
Similarily, other models including inhomogeneous Markov models, permuted Markov models, Bayesian networks, homogeneous Markov models, ZOOPS models, and hidden Markov models can be created using the \TrainSMFactory~and the \HMMFactory, respectively.

Given some model \lstinline+pwm+, we can directly infer the model parameters based on some data set \lstinline+ds+ using the \lstinline+train+ method.
\addtocounter{off}{2}
\code{0}
After the model has been trained, it can be used for scoring sequences using the \lstinline+getLogProbFor+ methods. More information about the interface \TrainSM~can be found in section \ref{tsm}.

We can build a classifier based on a set of \TrainSM s, e.g. two PWM models.
\renewcommand{\codefile}{\defaultcodefile}
\setcounter{off}{554}
\code{0}

\subsection{Further statistical models and classifiers}

Sometimes, we like to learn statistical models by other learning principles that require numerical optimization. For this purpose, Jstacs provides the interface \DiffSM~and the factory \DiffSMFactory~in close analogy to \TrainSM~and \TrainSMFactory~(cf. \ref{dsm}). Creating a classifier using two PWM models and the maximum supervised posterior learning principle can be accomplished by calling
\renewcommand{\codefile}{recipes/CreateMSPClassifier.java}
\setcounter{off}{49}
\code{0}
\addtocounter{off}{2}
\code{0}
\addtocounter{off}{2}
\code{0}

\subsection{Using classifiers}

\renewcommand{\codefile}{recipes/TrainClassifier.java}
As we have seen in the previous subsections, we can build classifiers based on statistical models. The main functionality is predicting the class label of a sequence and assessing the performance of a classifier. For these tasks, Jstacs provides the methods \lstinline+classify+ and \lstinline+evaluate+, respectively.

For classifying a sequence, we call
\setcounter{off}{58}
\code{0}
on a trained classifier. The method returns numerical class labels in the same order as data is provided for training starting from index $0$.
%The method returns values greater than or equal to 0 and less than the specified number of classes. 

For evaluating the performance of a classifier, we need to compute some performance measures. For convenience, Jstacs provides the possibility of getting a bunch of standard measures including point measures and areas under curves (cf. \ref{Performance}). Based on such measures, we can determine the performance of the classifier.
\addtocounter{off}{5}
\code{1}
Here, \lstinline+true+ indicates that an \lstinline+Exception+ should be thrown if a measure could not be computed, and \lstinline+data+ is an array of data sets, where the index within this array encodes the class.

For assessing the performance of a classifier using some repeated procedure of training and testing, Jstacs provides the class \ClassifierAssessment~(cf. \ref{Assessment}).

\renewcommand{\codefile}{\defaultcodefile}
\setcounter{off}{1}