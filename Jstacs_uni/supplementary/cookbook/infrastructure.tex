In the early days of Jstacs, we stored models, classifiers, and other Jstacs objects using the standard serialization of Java. However, this mechanism made it impossible to load objects of earlier versions of a class and the files where not human-readable. Hence, we started to create a facility for storing objects to XML representations. In the current version of Jstacs, this is accomplished by an interface \Storable~for objects that can be converted to and from their XML representation, and a class \XMLParser~that can handle such \Storable s, \Singleton s, Strings, Classes, primitives, and arrays thereof. In the first sub-section, we give examples how to use the \XMLParser.

Another problem we wanted to handle has been the documentation of (external) parameters of models, classifiers, or other classes. Although documentation exists in the Javadocs, these are inaccessible from the code. Hence, we created classes for the documentation of parameters and sets of parameters, namely the subclasses of \Parameter~and \ParameterSet. A \Parameter~at least provides the name of and a comment on the parameter that is described. In sub-classes, other values are also available like, for instance, the set or a range of allowed values. Such a description of parameters allows for manifold generic convenience applications. Current examples are the \link{parameters}{ParameterSetTagger}, which facilitates the documentation of command line arguments on basis of a \ParameterSet, or the \link{utils/galaxy}{GalaxyAdaptor}, which allows for an easy integration of Jstacs applications into the Galaxy webserver. We give examples for the use and creation of \Parameter s and \ParameterSet s in the second sub-section.

Finally, the same problem also occurrs for the results of computations. With a generic documentation, these results can be displayed together with some annotation in a way that is appropriate for the current application. In Jstacs, we use \Result s and \ResultSet s for this purpose, and we show how to use these in the third sub-section.

\subsection{XMLParser}\label{sec:xmlparser}
In the following examples, let \lstinline+buffer+ be some \lstinline+StringBuffer+. 
All kinds of primitives or \Storable s are appended to an existing \lstinline+StringBuffer+ surrounded by the specified XML tags by the static method \lstinline+appendObjectWithTags+ of \XMLParser. For example, the following two lines append an integer with the value $5$ using the tag \lstinline+integer+, and a \lstinline+String+ with the tag \lstinline+foo+:
\addtocounter{off}{9}
\code{3}
If we assume that \lstinline+buffer+ was an empty \lstinline+StringBuffer+ before appending these two elements, the resulting XML text will be
\begin{lstlisting}[language=XML]
<integer>5</integer>
<foo>hallo welt</foo>
\end{lstlisting}
In exactly the same manner, we can append XML representations of arrays of primitives, for example a two-dimensional array of \lstinline+double+ s
\addtocounter{off}{6}
\code{1}
or complete Jstacs models that implement the \Storable~interface
\addtocounter{off}{4}
\code{1}
or even arrays of \Storable s:
\addtocounter{off}{4}
\code{1}
The interface \Storable~only defines two things: first, an implementing class must provide a public method \lstinline+toXML()+ that returns the XML representation of this class as a \lstinline+StringBuffer+, and second, it must provide a constructor that takes a single \lstinline+StringBuffer+ as its argument and re-creates an object out of this representation. The only exception from this rule are singletons, i.e., classes that implement the \Singleton~interface.

Of course, you can use the \lstinline+appendObjectWithTags+ method of the \XMLParser~inside the \lstinline+toXML+ method. By this means, it is possible to break down the conversion of complex models into smaller pieces if the building-blocks of a model are also \Storable s.

In analogy to storing objects, the \XMLParser~also provides facilities for loading primitives and \Storable s from their XML representation. These can also be used in the constructor according to the \Storable~interface. For example, we can load the value of the integer, we stored a few lines ago by calling
\addtocounter{off}{4}
\code{0}
where the second argument of \lstinline+extractObjectForTags+ is the tag surrounding the value and, of course, must be identical to the tag we specified when storing the value. Since \lstinline+extractObjectForTags+ is a generic method, we must explicitly cast the returned value to an \lstinline+Integer+. As an alternative, we can also specify the class of the return type as a third argument like in the following example
\addtocounter{off}{3}
\code{0}
Here, we load the two-dimensional array of \lstinline+double+s that we stored a few lines ago.
In perfect analogy, we can also load a single instance of a class implementing \Storable
\addtocounter{off}{3}
\code{0}
where in this case we again specify the class of the return type in the third argument, or arrays of \Storable
\addtocounter{off}{3}
\code{0}

Of course, we can also specify the concrete sub-class of \Storable~for an array, if all instances are of the same class like in the following example:
\addtocounter{off}{3}
\code{2}

\subsection{Parameters \& ParameterSets}

Parameters in Jstacs are represented by different sub-classes of \Parameter, which define different types of parameters. Parameters that take primitives or strings as values are defined by the class \SimpleParameter, parameters that accept values from some \lstinline+enum+ type are defined by \EnumParameter, parameters where the user can select from a number of predefined values are defined by \SelectionParameter, parameters that represent a file argument are defined by \FileParameter, and parameters that represent a range of values are represented by \RangeParameter.
In the following, we give some examples for the creation of parameter objects. Let us assume, we want to define a parameter for the length of the sequences accepted by some model. The maximum sequence length this model can handle is $100$ and, of course, lengths cannot be negative. We create such a parameter object by the following lines of code:
\addtocounter{off}{9}
\code{0}
The first argument of the constructor defines the data type of the accepted values, which is an \lstinline+int+ in the example. The next two arguments are the name of and the comment for the parameter. The following boolean specifies if this parameter is required (\lstinline+true+) or optional (\lstinline+false+). The \NumberValidator~in the fifth argument allows for specifying the range of allowed values, which is $0$ to $100$ (inclusive) in the example. Finally, we define a default value for this parameter, which is $10$ in the example.
Similarly, we can define a \SimpleParameter~�for some optional parameter that takes strings as values by the following line:
\addtocounter{off}{1}
\code{0}
Again, the second and third arguments are the name and the comment, respectively.

We can define an \EnumParameter, which accept values from some \lstinline+enum+ type as follows
\addtocounter{off}{3}
\code{0}
where the first argument defines the class of the \lstinline+enum+ type, the second is the name of that collection of values, and the third argument again specifies if this parameter is required.

A \SelectionParameter~accepts values from a pre-defined collection of values. For instance, if we want the user to select from two double values $5.0$ and $5E6$, which are named ``small'' and ``large'', we can do so as follows:
\addtocounter{off}{3}
\code{0}

For the special case, where the user shall select the concrete implementation of an abstract class of interface, Jstacs provides a static convenience method \lstinline+getSelectionParameter+ in the class \SubclassFinder. This method requires the specification of the super-class of the \ParameterSet~that can be used to instantiate the implementations, the root package in which sub-classes or implementations shall be found, and, again, a name, a comment, and if this parameter is required.
For example, we can find all classes that can be instantiated by a sub-class of \SequenceScoringParameterSet~the package \lstinline+de+ and its sub-packages by calling
\addtocounter{off}{3}
\code{0}
The method returns a \SelectionParameter~from which a user can select the appropriate implementation.
Classes that can be found in this manner must implement an additional interface called \link{}{InstantiableFromParameterSet}. The main purpose of this interface is that implementing classes must provide a constructor that takes a \InstanceParameterSet~as its only argument in analogy to the constructor of \Storable~working on a \lstinline+StringBuffer+. \InstanceParameterSet s will be explained a few lines below.

As the name suggests, \ParameterSet s represent sets of such parameters. The most simple implementation of a \ParameterSet~is the \SimpleParameterSet, which can be created just from a number of \Parameter s like in the following example:
\addtocounter{off}{3}
\code{0}
Other \ParameterSet s are the \link{parameters}{ExpandableParameterSet} and \link{parameters}{ArrayParameterSet}, which can handle series of identical parameter types.

One special case of \ParameterSet s is the \InstanceParameterSet, which has several sub-classes that can be used to instantiate new Jstacs objects like statistical models or classifiers. If a new model, say an implementation of the \TrainSM~interface, shall be found via the \SubclassFinder, or its parameters shall be set in a command line program using the \link{parameters}{ParameterSetTagger} or in Galaxy, we need to create a new sub-class of \InstanceParameterSet~that represents all (external) parameters of this model. In this sub-class we must basically implement two methods: \lstinline+getInstanceName+ and \lstinline+getInstanceComment+ return the name of and a comment on the model class (i.e., in the example, the model we just implemented) that may be of help for a potential user. The constructor does the main work. By a call to the super-constructor, it initializes the list of parameters in this set and then adds the parameters of the model.
\addtocounter{off}{20}
%\code{18}
For implementations of the \TrainSM~interface we may also extend \link{parameters}{SequenceScoringParameterSet}, which already handles the \AlphabetContainer~and length of this model.

Not always do we have flat hierarchies of parameters. For instance, the choice of subsequent parameters may depend on the selection from some \SelectionParameter. For this purpose, Jstacs provides a sub-class of \Parameter~that only serves as a container for a \ParameterSet~and is called \link{parameters}{ParameterSetContainer}. Like other parameters, this container takes a name and a comment in its constructor, whereas the third argument is a \ParameterSet:
\addtocounter{off}{3}
\code{0}
Since such a \link{parameters}{ParameterSetContainer} can itself be part of another \ParameterSet, we can build hierarchies or trees of \Parameter s and \ParameterSet s. \link{parameters}{ParameterSetContainer}s are also used internally to create \SelectionParameter s from an array of \ParameterSet s, e.g., for \lstinline+getSelectionParameter+ in the \SubclassFinder.

\subsection{Results \& ResultSets}

We distinguish two types of \Result s, namely \NumericalResult s and \CategoricalResult s. The first are results containing numerical values, which can be aggregated, for instance averaged, while the latter are results of categorical values like strings or booleans. 
For example, we can create a \NumericalResult~containing a single \lstinline+double+ value by the following line
\addtocounter{off}{6}
\code{0}
where, in analogy to \Parameter s, the first and the second argument are the name of and a comment on the result, respectively.

Similarly, we create a \CategoricalResult~by the following line
\addtocounter{off}{3}
\code{0}
for a result that is a single \lstinline+boolean+ value.

As for \ParameterSet s, we can create sets of results using the class \ResultSet
\addtocounter{off}{3}
\code{0}
where we may also combine \NumericalResult s and \CategoricalResult s into a single set. Besides simple \ResultSet s, Jstacs comprises
\NumericalResultSet s for combining only \NumericalResult s, which can be created in complete analogy to \ResultSet s.

Jstacs also provides a special class for averaging \NumericalResult. 
This class is called \MeanResultSet, and computes the average and standard error of the corresponding values of a number of \NumericalResultSet s. The corresponding \NumericalResult s in the \NumericalResultSet~are identified by their name as specified upon creation.

We first create an empty \MeanResultSet~by calling its default constructor
\addtocounter{off}{3}
\code{0}
and subsequently add a number of \NumericalResultSet s to this \MeanResultSet.
\addtocounter{off}{2}
\code{3}
In the example, these are just 10 uniformly distributed random numbers.

Finally, we call the method \lstinline+getStatistics+ of \MeanResultSet~to obtain the mean and standard error of the previously added values.
\addtocounter{off}{4}
\code{0}
the result of this method is again returned as a \NumericalResultSet. In the example, it is just printed to standard out.