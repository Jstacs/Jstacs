\setcounter{off}{496}

For many tasks in Jstacs, especially for numerical parameter learning, we need numerical optimization techniques.

We start the description of numerical optimization in Jstacs with the definition of a function that depends on parameters to be optimized. The most simple way to define such a function is to extend the abstract class \NumericalDifferentiableFunction. In this abstract class, the gradient of the function is approximated by evaluating the function in an epsilon neightborhood around the current parameter values. Hence, a sub-class of \NumericalDifferentiableFunction~must only implement two methods. The first returns the number of parameters, and the second returns the function value for given parameter values.

Let us assume that we want to optimize, i.e., minimize, a function $f(x_1,x_2)=x_1^2 + x_2^2$. This functions depends on two parameters. Hence, we implement
\addtocounter{off}{6}
\code{2}
And we implement the method returning the function value as
\addtocounter{off}{5}
\code{2}
Now we are set to start a numerical optimization.

However, often we can derive the gradient analytically, which often yield a more efficient numerical optimization. Hence, the abstract class \DifferentiableFunction~allows to implement the computation of the gradient explicitly. To this end, we extend \DifferentiableFunction~and implement an additional method that computes the gradient:
\addtocounter{off}{19}
\code{2}

Now we can start a numerical optimization. To this end, we need to specify a \TerminationCondition~that determines when to stop the iterations of the optimization. For example, such a \TerminationCondition~may stop the optimization, if the difference of successive function evaluations does not exceed a given threshold:
\addtocounter{off}{7}
\code{0}
In this example, this threshold is set to $10^{-6}$. Another option would be to stop after $100$ iterations:
\addtocounter{off}{1}
\code{0}

And we may also combine several \TerminationCondition~in a \CombinedCondition:
\addtocounter{off}{3}
\code{0}
The first parameter ($2$) specifies that the \CombinedCondition~only allows to continue to the next iteration if both of the supplied \TerminationCondition s do so. 

For the numerical optimization, we create initial parameters and start the optimization:
\addtocounter{off}{3}
\code{1}
The arguments of this static method have the following meaning:
The first argument defines the technique for the optimization. In the example, we set this to the quasi-Newton method of Broyden, Fletcher, Goldfarb, and Shanno. As an alternative, Jstacs offers other quasi-Newton method including limited-memory variants, different conjugate gradients approaches, and steepest descent.
The second argument is the \DifferentiableFunction, the third are the initial parameter values, the fourth is the \TerminationCondition, and the fifth is the threshold on the difference of function values during the line search. Jstacs uses Brent's method, which is a combination of quadratic interpolation and golden ratio, for the line search. The sixth argument is the initial step size during the line search, and the last argument is an \lstinline+OutputStream+ to which output of the optimization is written. Here, we specify \lstinline+System.out+. If this argument is \lstinline+null+, output is suppressed.

After the optimization has finished, the supplied parameter array (\lstinline+parameters+ in the example) contains the optimal parameter values.

