In Jstacs, data is organized at three levels:
\begin{itemize}
  \item \Alphabet s for defining single symbols, and \AlphabetContainer s for defining aggregate alphabets,
  \item \Sequence s for defining sequences of symbols over a given alphabet,
  \item \DataSet s for defining sets of sequences.
\end{itemize}
Sequences are implemented as an array of numerical values. In case of discrete sequences over some symbolic alphabet,
the symbols are mapped to contiguous discrete values starting at $0$, which can be mapped back to the original symbols
using the alphabet. This mapping is also used for the \lstinline{toString()} method, e.g., for printing a sequence. The actual 
data type, i.e. byte, short, or integer, used to represented the symbols is chosen internally depending on the size of 
the alphabet. \Alphabet s, \Sequence s, and \DataSet s are immutable for reasons of security and data consistency. That means, an instance of those
classes cannot be modified once it has been created.

\subsection{Alphabets}

Since Jstacs is tailored at sequence analysis in bioinformatics, the most prominent alphabet is the \DNAAlphabet, which is a singleton instance that can be accessed by:
\addtocounter{off}{155}
\code{0}
For general discrete alphabets, i.e., any kind of categorical data, you can use a \DiscreteAlphabet. Such an alphabet can be constructed in case-sensitive and insensitive variants (first argument) using a list of symbols.
In this example, we create a case-sensitive alphabet with symbols "W", "S", "w", and "x":
\addtocounter{off}{3}
\code{0}
If you rather want to define an alphabet over contiguous discrete numerical values, you can do so by calling a constructor that takes
the minimum and maximum value of the desired range, and defines the alphabet as all integer values between minimum and
maximum (inclusive). For example, to create a discrete alphabet over the values from $3$ to $10$, you can call
\addtocounter{off}{3}
\code{0}
Continuous alphabets are defined over all reals (minus infinity to infinity) by default (see first line in the following example).
However, if you want to define the continuous alphabet over a specific interval, you can specify the maximum and the minimum value of that interval. In the example, we define a continuous alphabet spanning all reals between $0$ and $100$:
\addtocounter{off}{3}
\code{1}

For the DNA alphabet, each symbols has a complementary counterpart. Since in some cases, a similar complementarity
can also be defined for symbols other than DNA-nucleotides (e.g., for RNA sequences containing U instead of T), Jstacs
allows to define generic complementable alphabets. These allow for example the generation of reverse complementary sequences
out of an existing sequence. Here, we define a binary alphabet of symbols ``A'' and ``B'', where ``A'' is the complement of ``B'' and vice versa.
\addtocounter{off}{4}
\code{0}
The first parameter again defines if this alphabet is case-insensitive (which is the case), the second parameter defines the symbols of the alphabet, and the third parameter specifies the index of the complementary symbol. For instance, the symbol at position $1$ (``B'') is set as the complement of the symbol at position 0 (``A'') by setting the $0$-th value of the integer array to $1$.

After the definition of single alphabets, we switch to the creation of aggregate alphabets. Almost everywhere in Jstacs, we use aggregate alphabets
to maintain generalizability. Since the aggregate alphabet containing only a \DNAAlphabet~is always the same, a singleton for such an \AlphabetContainer~is pre-defined:
\addtocounter{off}{3}
\code{0}
We can explicitly define an \AlphabetContainer~using a simple continuous alphabet by calling:
\addtocounter{off}{3}
\code{0}

Aggregate alphabets become interesting if we need different symbols at different positions of a sequence, or even a mixture of discrete and continuous values.
For example, we might want to represent sequences that consist of a DNA-nucleotide at the first position, some other discrete symbol at the second position, and  a real number stemming from some measurement at the third position. Using the \DNAAlphabet, the discrete \Alphabet, and the continuous \Alphabet~defined
above, we can define such an aggregate alphabet by calling
\addtocounter{off}{3}
\code{0}
To save memory, we can also re-use the same alphabet at different position of the aggregate alphabet. If we want to use a \DNAAlphabet~at positions $0$, $1$, and $3$, and a continuous alphabet at positions $2$, $4$, and $5$, we can use a constructor that takes the alphabets as the first argument and the assignment to the positions as the second argument:
\addtocounter{off}{3}
\code{0}
The alphabets are assigned to specific positions by their index in the array of the first argument.

\subsection{Sequences}

Single sequences can be created from an \AlphabetContainer~and a string. However, in most cases, we load the data from some file, which will be explained
in the next sub-section.
For creating a DNA sequence, we use a \DNAAlphabet~like the one defined above and a string over the DNA alphabet:
\addtocounter{off}{3}
\code{0}
In a similar manner, we define a continuous sequence. In this case, a single value is represented by more than one letter in the string. Hence, 
we need to define a delimiter between the values as a third argument, which is a space in the example.
\addtocounter{off}{3}
\code{0}
We can also create sequences over the mixed alphabet defined above. In the example, the single values are delimited by a ``;''.
\addtocounter{off}{3}
\code{0}
For very large amounts of data or very long sequences, even the representation of symbols by byte values can be too memory-consuming. Hence, 
Jstacs also offers a representation of DNA sequences in a sparse encoding as bits of long values. You can create such a \SparseSequence~from a \DNAAlphabet~
and a string:
\addtocounter{off}{3}
\code{0}
However, the reduced memory footprint comes at the expense of a slightly increased runtime for accessing symbols of a \SparseSequence. Hence, it is not the default representation in Jstacs.

After we learned how to create sequences, we now want to work with them. First of all, you can obtain the length of a sequence
from its \lstinline+getLength()+ method:
\addtocounter{off}{3}
\code{0}
Since on the abstract level of \Sequence~we do not distinguish between discrete and continuous sequences (and we also may have mixed sequences), there are
two alternative methods to obtain one element of a sequence regardless of its content. With the first method, we can obtain the discrete value at a certain position (2 in the example):
\addtocounter{off}{3}
\code{0}
If the \Sequence~contains a continuous value at this position, it is discretized by default by returning the distance to the minimum value of the continuous alphabet
at this position casted to an integer. If the \Sequence~contains a discrete value, that value is just returned in the encoding according to the \AlphabetContainer.
In a similar manner, we can obtain the continuous value at a position (5 in the example)
\addtocounter{off}{1}
\code{0}
where discrete values are just casted to \lstinline+double+s.

We can obtain a sub-sequence of a \Sequence~using the method \lstinline+getSubSequence(int,int)+, where the first parameter is the start position within the sequence, counting from $0$, and the second
parameter is the length of the extracted sub-sequence. So the following line of code would extract a sub-sequence of length $3$ starting at position $2$ of the original sequence or, stated differently,
we skip the first two elements, extract the following three elements, and again skip everything after position $4$.
\addtocounter{off}{3}
\code{0}
Since \Sequence s in Jstacs are immutable, this method returns a new instance of \Sequence, which is assigned to a variable \lstinline+contSub+ in the example. Hence, in cases where you need
the same sub-sequences frequently in your code, for example in a ZOOPS-model or other models using sliding windows on a \Sequence, we recommend to precompute these sub-sequences and store them in some auxiliary data structure in order to invest runtime in computations rather than garbage collection. Internally, sub-sequences only hold a reference on the original sequences and the start position and length within that sequence to keep the memory overhead of sub-sequences at a low level.

For \Sequence s defined over a \ComplementableDiscreteAlphabet~like the \DNAAlphabet, we can also obtain the (reverse) complement of a sequence. For example, to create
the reverse complementary sequence of a complete sequence, we call
\addtocounter{off}{3}
\code{0}
For the complement of a sub-sequence of length $6$ starting at position $3$ of the original sequence, we use
\addtocounter{off}{3}
\code{0}

For some analyses, for instance permutation tests or for estimating false-positive rates of predictions, it is useful to create permuted variants of an original sequence.
To this end, Jstacs provides a class \PermutedSequence~that creates a randomly permuted variant using the constructor
\addtocounter{off}{3}
\code{0}
or a user-defined permutation by an alternative constructor. In the randomized variant, the positions of the original sequence are permuted independently of each other, which means that higher order properties of the sequence like di-nucleotide content are not preserved. If you want to create sequences with similar higher-order properties, have a look at the \lstinline+emitSample()+ method of \HomogeneousModel.

Often, we want to add additional annotations to a sequence, for instance the occurrences of some binding motif, start and end positions of introns, or just the species a sequence is stemming from. To this end, Jstacs provides a number of \SequenceAnnotation s that can be added to a \Sequence~(or read from a FastA-file as we will see later).
For instance, we can add the annotation for binding site of a motif called ``new motif'' of length $5$ starting at position $3$ of the forward strand of sequence \lstinline+dnaSeq+ using the \lstinline+annotate+ method of that sequence:
\addtocounter{off}{3}
\code{0}
Again, this method creates a new \Sequence~object due to \Sequence s being immutable.
After we added several \SequenceAnnotation s to a \Sequence, we can obtain all those annotations by calling
\addtocounter{off}{3}
\code{0}
For retrieving annotations of a specific type, we can use the method \lstinline+getSequenceAnnotationByType+
\addtocounter{off}{3}
\code{0}
to, for instance, obtain the first (index $0$) annotation of type ``Motif''.

\subsection{DataSets}
In most cases, we want to load \Sequence s from some FastA or plain text file instead of creating \Sequence s manually from strings. In Jstacs, collections of \Sequence s are represented by
\DataSet s. The class \DataSet~(and \DNADataSet) provide constructors that work on a file or the path to a file, and parse the contents of the file to a \DataSet, i.e. a collection of \Sequence s.

The most simple case is to create a \DNADataSet~from a FastA file. To do so, we call the constructor of \DNADataSet with the (absolute or relative) path to the FastA file:
\addtocounter{off}{3}
\code{0}
For other file formats and types of \Sequence s, \DataSet~provides another constructor that works on the \AlphabetContainer for the data in the file, a \StringExtractor~that handles the extraction of
the strings representing single sequences and skipping comment lines, and a delimiter between the elements of a sequence. Hence, the \StringExtractor, a \SparseStringExtractor~in the example, requires the specification of the path to the file and the symbol that indicates comment line. For example, if we want to create a sample of continuous sequences stored in a tab-separated plain text file ``myfile.tab'', we use the \AlphabetContainer~with a continuous \Alphabet~from above, a \StringExtractor~with a hash as the comment symbol, and a tab as the delimiter:
\addtocounter{off}{3}
\code{0}
The \SparseStringExtractor~is tailored to files containing many sequences, and reads the file line by line, where each line is converted to a \Sequence~and discarded before the next line is parsed.

Since \SparseSequence s are not one of the default representations of sequence in Jstacs (see above), these are not created by the constructors of \DataSet~or \DNADataSet. However, the class \SparseSequence~provides a method \lstinline+getSample+ that takes the same arguments as the constructor of \DataSet, for example
\addtocounter{off}{3}
\code{0}
for reading DNA sequences from a FastA file, and returns a \DataSet~containing \SparseSequence s.

After we successfully created a \DataSet, we want to access and use the \Sequence s within this \DataSet. We retrieve a \Sequence~of a \DataSet~using the method \lstinline+getElementAt(int)+. For instance,
we get the fifth \Sequence~of \lstinline+dnaSample+ by calling
\addtocounter{off}{3}
\code{0}
We can also request the number of \Sequence s in a \DataSet~by the method \lstinline+getNumberOfElements()+ and use this information, for instance, to
iterate over all \Sequence s. In the example, we just print the retrieved \Sequence s to standard out
\addtocounter{off}{2}
\code{2}
where the \Sequence s are printed in their original alphabet since their \lstinline+toString()+ method is overridden accordingly.

As an alternative to the iteration by explicit calls to these methods, \DataSet~also implements the \lstinline+Iterable+ interface, which facilitates the Java variant
of foreach-loops as in the following example:
\addtocounter{off}{5}
\code{2}
Here, we just print the length of each \Sequence~in \lstinline+contSample+ to standard out.

We can also apply some of the sequence-level operations to all \Sequence s of a \DataSet, and obtain a new \DataSet~containing the modified sequences. For example, we get a \DataSet~containing 
the sub-sequences of length $10$ starting at position $3$ of each sequence by calling
\addtocounter{off}{5}
\code{0}
a \DataSet~of all suffixes starting at position $7$ from
\addtocounter{off}{3}
\code{0}
or a \DataSet~containing all reverse complementary \Sequence s using
\addtocounter{off}{3}
\code{0}

For cross-validation experiments, hold-out samplings, or similar procedures (cf. \href{\APIhome de/jstacs/classifier/assessment/package-summary.html}{package assessment}), it is useful to partition a sample randomly. \DataSet s in Jstacs support two types of partitionings.
The first is to partition a \DataSet~into $k$ almost equally sized parts. What is ``equally sized'' can either be determined by the number of sequences or by the number of symbols of all sequences in a sample. Both measures are supported by Jstacs. 

The second partitioning method creates partitions of a user-defined fraction of the original sample. For example, we partition the \DataSet~\lstinline+dnaSample+ into five equally sized parts according to the number of sequences in that \DataSet~by calling
\addtocounter{off}{3}
\code{0}
and we partition the same sample into parts containing $10$, $20$, and $70$ percent of the symbols of the original \DataSet~by calling
\addtocounter{off}{1}
\code{0}
In both cases, the \Sequence s in the \DataSet~are partitioned as atomic elements. That means, a \Sequence~is not cut into several parts to obtain exactly equally sized parts, but the size of a part may slightly (depending on the number of sequences and lengths of those sequences) differ from the specified percentages.

To create a new \DataSet~that contains all sub-sequences of a user-defined length of the original \Sequence s, we can use another constructor of \DataSet. The sub-sequences are extracted in the same manner as we would do by shifting a sliding window over each sequence, extracting the sub-sequence under this window, and build a new \DataSet~of the extracted sub-sequences.
For instance, we obtain a \DataSet~with all sub-sequences of length $8$ using
\addtocounter{off}{3}
\code{0}

In the previous sub-section, we learned how to add \SequenceAnnotation s to a \Sequence. Often, we want to use the annotation that is already present in an input file, for example
the comment line of a FastA file. We can do so by specifying a \SequenceAnnotationParser~in the constructor of the \DataSet. The simplest type of \SequenceAnnotationParser~�is the \SimpleSequenceAnnotationParser, which just extracts the complete comment line preceding a sequence.
\addtocounter{off}{3}
\code{0}
Although the specification of the parser is quite simple, the extraction of the comment line as a string is a bit lengthy. We first obtain the \Sequence~from the \DataSet, get the annotation of that sequence, obtain the first comment, called ``result'' in the hierarchy of Jstacs (you see in section~\ref{sec:xmlparres}, why), and convert the corresponding result object to a string.
\addtocounter{off}{3}
\code{0}

If your comment line is defined in a ``key-value'' format with some generic delimiter between entries, you can Jstacs let parse the entries to distinct annotations. For instance, if the comment line has some format \lstinline+key1=value1; key2=value2;...+, we can parse that comment line using the \SplitSequenceAnnotationParser. This parser only requires the specification of the delimiter between key and value (``=`` in the example) and the delimiter between different entries (``;'' in the example). Like before, we instantiate a \SplitSequenceAnnotationParser~as the last argument of the \DNADataSet~constructor:
\addtocounter{off}{3}
\code{0}
We can now access all parsed annotations by the \lstinline+getAnnotation()+ method
\addtocounter{off}{3}
\code{0}
or, for instance, the \lstinline+getSequenceAnnotationByType+ introduced in the previous section, where the type corresponds to the key in the comment line, and the identifier of the retrieved \SequenceAnnotation~is identical to the value for that key in the comment line.

Jstacs only supports FastA and plain text files directly. However, you can access other formats or even databases like Genbank using an adaptor to BioJava.

For example, we can use BioJava to load two sequences from Genbank.
\addtocounter{off}{3}
\code{6}
As a result, we obtain a \lstinline+RichSequenceIterator+, which implements the \lstinline+SequenceIterator+ interface of BioJava. We can use a
\lstinline+SequenceIterator+ in an adaptor method to obtain a Jstacs \DataSet~including converted annotations:
\addtocounter{off}{7}
\code{0}
The second argument of the method allows for filtering for specific annotations using a BioJava \lstinline+FeatureFilter+.

Vice versa, we can convert a Jstacs \DataSet~to a BioJava \lstinline+SequenceIterator+ by an analogous adaptor method:
\addtocounter{off}{2}
\code{0}

By means of these two methods, we can use all BioJava facilities for loading and storing data from and to diverse file formats and loading data from data bases in our Jstacs applications.